%travail_abstract_factory
\subsection{Fabrique abstraite}
La dernière partie du projet consistait à créer plusieurs familles de soldats historiquement cohérentes, c'est-à-dire que pour une famille donnée, le type d'armement devait être conforme à son époque.

Pour le client, la notion d'époque vis-à-vis des armes doit rester transparente. En effet, on veut associer les bonnes armes aux soldats construits par le client sans qu'il ait à se soucier de la cohérence de son habillement. Il n'aura donc pas à spécifier lui même les classes concrètes correspondant aux armes d'une époque ou d'une autre pour habiller son soldat. Ainsi, le client n'aura pas accès au processus de construction de son équipement (telle arme de défense pour telle époque, telle arme d'attaque pour telle époque), tout se fera automatiquement.

Pour réaliser cette demande, nous avons utilisé le pattern \emph{Fabrique abstraite}. Nous avons donc créé une classe \emph{AbstractFactory} comportant les méthodes \emph{setDefensiveWeapon(Soldier s)} et \emph{setOffensiveWeapon(Soldier s)}, qui permettent d'affecter à un soldat donné une arme défensive/offensive conforme à l'époque. Ces méthodes se chargeront alors de la construction de l'équipement souhaité, en toute conformité avec l'époque. L'implémentation de ces deux méthodes se trouve dans les classes \emph{MiddleAgeFactory} et \emph{ScienceFictionFactory}, qui représentent deux époques différentes. L'armement pourra ainsi être construit différemment en fonction de l'époque (par exemple les soldats de l'époque moderne seront équipés de lasers).
Nous avons également rajouté des méthodes dans la fabrique permettant de construire des soldats de type différent (soldat à pied, soldat monté).

Ainsi, grâce au pattern \emph{Fabrique abstraite}, nous pouvons garantir le maintient de la cohérence de nos équipements et de nos soldats en fonction de l'époque.