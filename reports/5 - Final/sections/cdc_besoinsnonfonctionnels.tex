\section{Besoins non-fonctionnels}

\subsection{Refactoring}\label{refactoring} 

Le refactoring consiste à retravailler un code source dans le but d'améliorer sa lisibilité et son efficacité, et de simplifier sa maintenance.
L'introduction de nouveaux patterns induit le besoin de refactoriser souvent le code afin de simplifier
le codage et la compréhension.
En plus d'un simple nettoyage, cela nous amène à vérifier que notre architecture répond toujours aux
objectifs fixés.

L'objectif est bien sûr d'obtenir un gain de clarté, de lisibilité, de maintenabilité, et probablement de performances. Nous pouvons ainsi continuer l'ajout de fonctions sur une base saine.

\subsection{Transparence pour le client}

Nous pensons qu'un des objectifs du génie logiciel est de tenter d'obtenir une architecture telle que l'apport
de nouvelles fonctionnalités se fasse de manière transparente pour le client. Nous pourrions donc ajouter des
contraintes ou des fonctions sans que le \emph{main} ou les tests unitaires soient dégradés.

\subsection{Flexibilité}

La flexibilité d'une architecture est une pierre angulaire pour la maintenabilité d'une application.
Dans cette optique, nous devons garder notre code le plus simple possible, mais également penser aux contraintes
induites par l'utilisation des patterns.